\kafkasection{Scaling Behavior (4 Points)
}
\kafkasubsection{Near critical points, the self-similarity under rescaling leads to characteristic power-law singularities. These dependences may be disguised, however, by less-singular corrections to scaling. An experiment measures the
susceptibility $\chi(T)$ in a magnet for temperatures T slightly above the ferromagnetic transition temperature Tc. They find their data is fit well by the form
\begin{equation}
    \chi(T) = A(T-T_c)^{-1.25}+ B + C(T-T_c) + D(T-T_c)^{1.77}
\end{equation}
Assuming this is the correct dependence near Tc, what is the critical exponent $\gamma$
}
The critical exponent is the one belonging to the term that has the leading behavior. We can see that for the $D$ term, a temperature around the critical temperature is still a small number. This is true with $B$ and $C$ as well since their exponent is 1.
However, the exponent for $A$ is negative. This means that when $T\rightarrow T_c$, we get a diverging term. This means that:
\begin{equation}
    \boxed{\gamma = -1.25}
\end{equation}
\kafkasubsection{
The pair correlation function $C(r,T) = \braket{\sigma_i\sigma_{i+r}}$ is measured in another,
three-dimensional system just above $T_c$. It is found to be spherically symmetric, and of the form
\begin{equation}
    C(r,T) = r^{-1.026} G(r(T-T_c)^{0.59})
\end{equation}
Where the function $G(x)$ is found to be roughly $exp(-x)$. What is the critical exponent $\nu$?
}
The critical exponent comes from the how the correlation length diverges with $(T-T_c)^{-\nu}$. Since $0.59$ is the exponent on this factor in our correlation function, we have:
\begin{equation}
    \boxed{\nu=0.59}
\end{equation}