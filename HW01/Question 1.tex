\begin{figure}
    \centering
    \includegraphics[width=0.8\linewidth]{image.png}
    \caption{Caption}
    \label{fig:placeholder}
\end{figure}
\kafkasection{
So far we have considered the ferromagnetic Ising model for which the energy of interaction between
two nearest neighbor spins is $J > 0$. Hence, all spins are parallel in the ground state of the
ferromagnetic Ising model. In contrast, if $J < 0$, nearest neighbor spins must be antiparallel to
minimize their energy of interaction.
}
\kafkasubsection{Sketch the ground state of the one-dimensional antiferromagnetic Ising model. Then do the
same for the antiferromagnetic Ising model on a square lattice. What is the value of M for
the ground state of an Ising antiferromagnet?}
The ground state is simply the least energy configuration. This is when we have alternating spins as nearest neighbors.
\begin{figure}[H]
    \centering
    \includegraphics[width=0.5\linewidth]{Q1A.png}
    \caption{sketch of the ground state of the 1-D antiferromagnetic Ising model}
\end{figure}
\begin{figure}[H]
    \centering
    \includegraphics[width=0.5\linewidth]{Q1B.png}
    \caption{sketch of the ground state of the antiferromagnetic Ising model}
\end{figure}
The value of $M$ is then $M=0$ since the magnetization is zero. 
\kafkasubsection{Use Program \texttt{IsingAnitferromagnetSquareLattice} to simulate the antiferromagnetic Ising
model on a square lattice at various temperatures and describe its qualitative behavior. Does
the system have a phase transition at $T > 0$? Does the value of M show evidence of a phase
transition?}
For very low values of the temperature ($~0.1K$). We get a checkerboard of spin states. Slowly increasing the temperature we get small perturbations but a general checkerboard look. At around $T = 2K$ the checkerboard goes away with higher temperatures complete chaos and spin states flipping. 
This means that there is a phase transition at $T>0$ where the value of $M$ does not show any evidence of it. This is because $M$ is largely unchanged by changing the temperature around the critical temperature.
\kafkasubsection{
In addition to the usual thermodynamic quantities the program calculates the staggered magnetization and the staggered susceptibility. The staggered magnetization is calculated by considering the square lattice as a checkerboard with black and red sites so that each black site has four red sites as nearest neighbors and vice versa. The staggered magnetization is calculated from $\sum c_is_i$ where $c_i = +1$ for a black site and $c_i = -1$ for a red site. Describe the behavior of these quantities and compare them to the behavior of M and $\chi$ for the ferromagnetic Ising model.
}
The staggered susceptibility is relatively low for high and low T but spikes at the critical temperature. The staggered magnetization is high for low T and then lowers to zero at and after the critical temperature.The behaviour of $M$ in the ferromagnetic model is then anaolgous to the staggered magnetization behavior and the behavior of $\chi$ in ferromagnetic is anaolgous to the staggered susceptibility.
\kafkasubsection{
Consider the Ising antiferromagnetic model on a hexagonal lattice (see Figure 5.11), for which
each spin has six nearest neighbors. The ground state in this case is not unique because of
frustration (see Figure 5.12). Convince yourself that there are multiple ground states. Is the entropy zero or nonzero at T = 0? Use Program \texttt{IsingAntiferromagnetHexagonalLattice}
to simulate the antiferromagnetic Ising model on a hexagonal lattice at various temperatures
and describe its qualitative behavior. This system does not have a phase transition for $T > 0.$ Are your results consistent with this behavior?
}
Because of the "frustrated" states that occur when you try to minimize the energy, the entropy is a nonzero value since at $T=0$ there are more multiple ground states of frustrated spins. Testing for different T we don't get radically different behavior in the spin flips. It is a sea of flips no matter the temperature. This is consistent with no phase transitions.
