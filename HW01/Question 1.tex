\begin{figure}
    \centering
    \includegraphics[width=0.8\linewidth]{image.png}
    \caption{Caption}
    \label{fig:placeholder}
\end{figure}
\kafkasection{
So far we have considered the ferromagnetic Ising model for which the energy of interaction between
two nearest neighbor spins is $J > 0$. Hence, all spins are parallel in the ground state of the
ferromagnetic Ising model. In contrast, if $J < 0$, nearest neighbor spins must be antiparallel to
minimize their energy of interaction.
}
\kafkasubsection{Sketch the ground state of the one-dimensional antiferromagnetic Ising model. Then do the
same for the antiferromagnetic Ising model on a square lattice. What is the value of M for
the ground state of an Ising antiferromagnet?}
\kafkasubsection{Use Program \texttt{IsingAnitferromagnetSquareLattice} to simulate the antiferromagnetic Ising
model on a square lattice at various temperatures and describe its qualitative behavior. Does
the system have a phase transition at T > 0? Does the value of M show evidence of a phase
transition?}
\kafkasubsection{
In addition to the usual thermodynamic quantities the program calculates the staggered magnetization and the staggered susceptibility. The staggered magnetization is calculated by considering the square lattice as a checkerboard with black and red sites so that each black site has four red sites as nearest neighbors and vice versa. The staggered magnetization is calculated from $\sum c_is_i$ where $c_i = +1$ for a black site and $c_i = −1$ for a red site. Describe the behavior of these quantities and compare them to the behavior of M and $\chi$ for the ferromagnetic Ising model.
}
\kafkasubsection{
Consider the Ising antiferromagnetic model on a hexagonal lattice (see Figure 5.11), for which
each spin has six nearest neighbors. The ground state in this case is not unique because of
frustration (see Figure 5.12). Convince yourself that there are multiple ground states. Is the entropy zero or nonzero at T = 0? Use Program \texttt{IsingAntiferromagnetHexagonalLattice}
to simulate the antiferromagnetic Ising model on a hexagonal lattice at various temperatures
and describe its qualitative behavior. This system does not have a phase transition for $T > 0.$ Are your results consistent with this behavior?
}
