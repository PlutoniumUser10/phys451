\kafkasection{Minima of free energy}
\kafkasubsection{To understand the meaning of the various solutions of (5.108), expand the free energy in
(5.126) about $m = 0$ with $H = 0$ and show that the form of $f(m)$ near the critical point (small m) is given by
\begin{equation}
    f(m) = a + b(1-\beta qJ)m^2 + cm^4
    \label{eq:fm}
\end{equation}
Determine $a, b, $ and $c$
}
\kafkasubsection{If $H$ is nonzero but small, show that there is an additional term $-mH$ in \ref{eq:fm}}
\kafkasubsection{Show that the minimum free energy for $T>T_c$ and $H=0$ is at $m=0$, and that $m = \pm m_0$ corresponds to a lower free energy for $T<T_c$.}
\kafkasubsection{Use program \texttt{IsingMeanField} to plot $f(m)$ as a function of $m$ for $T>T_c$ and $H=0$. For what value of $m$ does $f(m)$ have a minimum?}
\kafkasubsection{Plot $f(m)$ FOR $T=1$ and $H=0$. Where are the minima of $f(m)$? Do they have the same depth? If so, what is the meaning of this result?}
\kafkasubsection{Choose $H=0.5$ and $T=1$. Do the two minima have the same depth? The global minimum corresponds to the equilibrium or stable phase. If we quickly "flip" the field and let $H \rightarrow -0.5$, the minimum at $m\approx 1$ will become a local minimum. The system will remain in this local minimum for some time before it switches to the global minimum.}