%Math Stuff
\usepackage{amsmath} % many display options for math modes and such (VERY IMPORTANT)
%\usepackage{physics} % useful math symbols and macros (VERY IMPORTANT)
\usepackage{amssymb} % heck ton of math symbols and fonts (VERY IMPORTANT)
\usepackage{amsfonts} % Additional fonts, math symbols, and options for existing fonts (IMPORTANT)
\usepackage{mathtools} % lots of cool mathtools oh wait that is the name (IMPORTANT)
\usepackage{siunitx} % allows you to quickly define units within math mode (IMPORTANT)
\usepackage{array} % allows you to write piece-wise functions, matrices, and other cool things (IMPORTANT)
%\usepackage{txfonts} % defines times new roman as default text font and provides supporting math symbols
\usepackage{braket} % allows the use of detailed Dirac braket notation
\usepackage{cancel} % used to cancel terms (with X or diagonal line) while writing math (useful for assignments)
\usepackage{ulem} % underlines, typewriter font (for comp sci somewhat), and striking out (required for cancel package)
\usepackage{halloweenmath} % spooky
\usepackage{empheq} %Extension of amsmath for boxing equations or answers

%Jaynard's packages
\usepackage[table,xcdraw]{xcolor} 
\usepackage{amstext} %for \text macro
\newcolumntype{L}{>{$}l<{$}} % math-mode version of "l" column type
\usepackage{gensymb} %degrees lmao
\usepackage{pdfpages} %Allows to insert pdfs
\usepackage[f]{esvect} %unfuck the vector 
\usepackage{multicol} %Collumns 
\newcommand{\solar}{\odot} %Solar symbol
\newcommand{\earth}{\oplus} %earth symbol
\numberwithin{equation}{section} %turn 1.1 to 1.a
\newcommand{\topower}[1]{\times 10^{#1}} %quick sci not
\usepackage{matlab-prettifier} %showw mat lab better
\usepackage{mathrsfs} %Has fancy R's lmao
\usepackage{tcolorbox} %Colour boxes
\newcommand{\fratio}{\mathscr{R}} %Focal Ratio
\usepackage[export]{adjustbox}
\usepackage{physics2} %physics but better lmao
\usephysicsmodule{ab} %bracing
\usepackage{fixdif,derivative} %better derivatives
\DeclareMathOperator{\Lagr}{\mathcal{L}} %Lagrangian
\DeclareMathOperator{\Haml}{\mathcal{H}} %Hamiltonian
\newcommand{\evalat}{\biggr\rvert}%evaluate at for integration or partials

%Jaynard's Colors
\definecolor{KafkaMain}{HTML}{DDA0DD}
\definecolor{KafkaDark}{HTML}{8b008b}
\definecolor{KafkaDarker}{HTML}{301934}
\definecolor{KafkaLight}{HTML}{6f2da8}

%Formatting Stuff
%\usepackage[utf8]{inputenc} % the typesetting rules
\usepackage[left = 22mm, right = 22mm]{geometry} % flexible and easy interface to change page dimensions
\usepackage{graphicx} % provides additional options for figures
\usepackage{float} % allows you to tell latex no really I want the figure HERE (with H)
\usepackage{color} % provides coloring capabilities to everything
\usepackage{fancyhdr} % making fancy headers (obviously)
\usepackage{changepage} % allows to change page formatting in the middle of a document (rather than the same throughout) 
\usepackage{enumitem} % better control over enumerate and itemize
\usepackage[labelfont=bf]{caption} % altering options for captions
\usepackage{sidecap} % allows sideways captions for figures and tables
\usepackage{soul,xcolor} % can enhance striking, underlining, and highlighting (in non-math mode)


%Stuff for Tables
\usepackage{booktabs} % allows weird lines within tables
\usepackage{multirow}
\usepackage{multicol} % allows for multiple columns and rows that collapse into single columns
\usepackage{longtable} % allows a table to run over multiple pages
\usepackage{colortbl} % colourful tables!

%Stuff for Citing
\usepackage{apacite} % helps you apa cite
\usepackage[numbers]{natbib} % implements both author-year and numbered references
\usepackage{url} % allows options for url links
\usepackage[colorlinks, citecolor = black, urlcolor = black, linkcolor = black]{hyperref} %references (citation) options and hyperlink integration


%Chemistry Stuff
\usepackage[version=4]{mhchem} % write chemical equations within math mode
\usepackage{chemfig} % draw simple and complex chemical structures in latex

%Elijah Adams Created Commands
\newcommand{\Q}{\mathbb{Q}} %Mathematical sets
\newcommand{\R}{\mathbb{R}}
\newcommand{\C}{\mathbb{C}}
\newcommand{\Z}{\mathbb{Z}} 
\newcommand{\Curl}{\vec{\nabla}\times} % Good looking curl and div and laplace (because physics package ones look like shit)
\newcommand{\Div}{\vec{\nabla}\cdot}
\newcommand{\laplace}{\vec{\nabla}^2}
\newcommand{\yas}{\frac{1}{4\pi\epsilon_0}} % Useful for E&M stuff
\newcommand{\mas}{\frac{\mu_0}{4\pi}}
\newcommand{\ppdv}[2]{\frac{\partial^2 #1}{\partial #2^2}} % Twice partial derivatives

\setlength{\parindent}{4mm}
\definecolor{purple}{rgb}{0.84,0.71,0.85} % defining a colour for later, each rgb value is a ratio of r/255, g/255 and b/255 because 255 is the maximum value of rgb numbers
\newcolumntype{d}{>{\columncolor{purple}}c} 

%Kafka Section Header
\newcommand{\nocontentsline}[3]{}
\newcommand{\tocless}[2]{\bgroup\let\addcontentsline=\nocontentsline#1{#2}\egroup}
\newcommand{\kafkasection}[1]{
\begin{tcolorbox}[
colback=KafkaMain,
colframe=KafkaDarker,
colbacktitle=KafkaDark,
title={\protect \begin{minipage}[c]{6.75em}\vspace*{0.2em}\Large\textbf{Question}\normalsize\end{minipage}\begin{minipage}[c]{5em}\tocless\section{}\end{minipage}}]
    #1
\end{tcolorbox}\addcontentsline{toc}{section}{Question \thesection}}

%Kafka Subsection Header
\newcommand{\kafkasubsection}[1]{
\begin{tcolorbox}[
colback=KafkaMain,
colframe=KafkaDarker,
colbacktitle=KafkaDark,
title={\protect \begin{minipage}[c]{6.75em}\vspace*{0.2em}\Large\textbf{Question}\normalsize\end{minipage}\begin{minipage}[c]{5em}\tocless\subsection{}\end{minipage}}]
    #1
\end{tcolorbox}\addcontentsline{toc}{section}{Question \thesubsection}}

%Abs w/o 
\DeclarePairedDelimiter\abs{\lvert}{\rvert}%
\DeclarePairedDelimiter\norm{\lVert}{\rVert}%

% Swap the definition of \abs* and \norm*, so that \abs
% and \norm resizes the size of the brackets, and the 
% starred version does not.
\makeatletter
\let\oldabs\abs
\def\abs{\@ifstar{\oldabs}{\oldabs*}}
%
\let\oldnorm\norm
\def\norm{\@ifstar{\oldnorm}{\oldnorm*}}
\makeatother
